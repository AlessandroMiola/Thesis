\chapter{Conclusions}
\label{chpr:conclusions}
This thesis has tried to motivate the importance for Bitcoin to adopt technologies enhancing privacy in the incoming years. This would not merely improve the privacy of people transacting (which is indeed fundamental), but even strengthen its ability to serve as money. Indeed, though it cannot be recognized as a good unit of account, Bitcoin is a good store of value (it is durable, it can be reliably saved, stored with low costs and easily retrieved) and an excellent medium of exchange (it is easily portable, divisible, swappable, resistant to counterfeiting). Its greatest lack is that it is not that fungible and it is the case that fungibility is strictly linked to privacy.\\
At the same time, this work should have outlined the reasons why no privacy-based solutions have been soft-forked yet in Bitcoin at the time of writing; this is not certainly for a lack of proposals, quite the opposite. Developers have worked in this direction since long time, but cryptographic, privacy-based solutions are costly and require commitments, thus opening other issues.\\
Moreover, through confidential transactions \cite{Max15} we have explored some nice features of homomorphic encryption applied to commitment schemes, the interesting field of Zero-Knowledge proofs, a fancy variant of digital signature scheme and a clever solution for the communication of transaction amount, blinding factors and other user-selected metadata between participants in the transaction.\\
Confidential transactions basically hides each output amount through a Pedersen commitment to the amount and add a range proof ensuring that the amount does not overflow. The solution we have described builds each range proof through a Borromean ring signature. Among the strengths of confidential transactions, the fact that these can be possibly constructed without new cryptographic assumptions with respect to the main protocol, but relying on the hardness of the ECDLP (differently from some alternative schemes like Zcash's ones) and the substantial savings in terms of size and verification time with respect to previous solutions (which have made them sources of inspiration for privacy-based alt-coins like Monero, that has effectively implemented confidential transactions through ring signatures, RingCT \cite{RingCTMonero}). On the other hand, the solution suffers from the size of the range proof attached to each transaction output (and thus of the entire transaction) being too large.\\
These weaknesses have prevented confidential transactions to be soft-forked in Bitcoin up to now.\\
More recently, however, a new and more efficient solution to range proof construction \cite{Bulletproofs} has been proposed. Its name is Bulletproofs and it is likely worth studying: it could be the solution being effectively soft-forked in the future (it even naturally marries some older proposals) and effectively bringing consistent privacy in the protocol. Indeed, Bulletproofs is still well-suited for constructing efficient range proofs on committed value, but it also adds various optimizations.\\
At first, it provides aggregation of range proofs: it would be possible to prove that $m$ commitments lie in a given range by just providing additional O($\log m$) group elements with respect to a single proof, making it growing logarithmically with the number of transaction outputs. This would be already useful for confidential transactions by themselves as standard Bitcoin transactions have generally at least two outputs and it would even make it efficiently combine with Coinjoin \cite{Max13}. Additionally it could simultaneously double the range proof precision at marginal additional cost.\\
Then, it would allow batched verification of multiple Bulletproofs.\\ All of these enhancements with a total transaction size not so bigger than a standard transaction according to \cite{Bulletproofs,CT_eff}.\\
Moreover, confidential transactions are even beneficial for a newborn and promising cryptosystem called Mimblewimble \cite{MW, PoeMW}. Mimblewimble is still a transaction output based system (thus a Bitcoin-like blockchain system) which however implements confidential transactions from the beginning. At the time of writing it is already being built through Bulletproofs, thus inheriting its benefits. Moreover, Mimblewimble removes the need for the unlocking script because it allows to prove a transaction to commit to a Pedersen commitment to 0 just signing the transaction through the difference of the commitments to outputs and inputs. Other than this, it can benefit from transaction aggregation and enables the construction of a simplified blockchain where spent transactions can be pruned, thus improving scalability.