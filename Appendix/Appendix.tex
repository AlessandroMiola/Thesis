\chapter{Abstract algebra fundamentals}
\label{app:A}
This appendix is aimed at providing some fundamental notions of  algebra of sets and number theory at the basis of the considered public-key cryptosystem. Definitions are mainly taken from \cite{UnderstandingCrypto} and adapted when needed.
\section{Groups}
We start from the definition of a \textit{group}.
\begin{mydef} A group is a set of elements $\mathbb{G}$ together with an operation $\circ$ which combines two elements of $\mathbb{G}$. A group  satisfies the following properties:
	\begin{itemize}
		\item The group operation $\circ$ is closed: $\forall a, b \in \mathbb{G} \rightarrow a \circ b \in \mathbb{G}$.
		\item The group operation $\circ$ is associative: $\forall a, b, c \in \mathbb{G}  \rightarrow (a \circ b) \circ c = a \circ (b \circ c)$.
		\item Identity: $\exists e \in \mathbb{G} \ | \ \forall a \in \mathbb{G}, \ e \circ a = a \circ e = a$.
		\item Invertibility: $\forall a \in \mathbb{G}, \ \exists b \in \mathbb{G} \ | \ a \circ b = b \circ a = e$. This element is called inverse of a and it is commonly denoted either as $a^{-1}$ or $-a$, depending on the notation (multiplicative or additive).
		\item A group $\mathbb{G}$ is abelian (or commutative) if, furthermore, $\forall a, b \in \mathbb{G} \rightarrow a \circ b = b \circ a$.
	\end{itemize}
\end{mydef}
\label{defA1}
\noindent
Depending on whether we consider additive or multiplicative notation, the operation $\circ$ stands respectively for addition or multiplication.
\begin{myrem} The group operation $\circ$ is called group law of $\mathbb{G}$.
\end{myrem}
\begin{myrem} The number of elements in a group $\mathbb{G}$ is called group order (or cardinality). We denote it by $|\mathbb{G}|$.
\end{myrem}
\begin{myexample}
$(\mathbb{Z},+)$ is a group. Particularly, it forms an abelian group where $e = 0$ is the identity element, $b = -a$ is the inverse of an element $a \in \mathbb{Z}$.
\end{myexample}
\begin{myexample}
$(\mathbb{Z} \backslash \{0\},\cdot)$ is \textbf{not} a group. Particularly, $\nexists b = a^{-1}$ for an element $a \in \mathbb{Z}$ with the exception of the elements -1 and 1.
\end{myexample}
\begin{myexample}
$(\mathbb{Z}_m,+)$, where $\mathbb{Z}_m$ = $\{0,1,\dots,m-1\}$ and the operation is the addition modulo m, form a group (of order $m$) with the identity element $e = 0$. Every element $a$ has an inverse $b=-a$ such that $a + (-a) = 0$ mod m.
\end{myexample}
\begin{myrem}
This last example points out a straightforward, yet important, fact. By definition, the inverse must belong to the group $\rightarrow$ $b=m-a$ is the inverse of any group element $a$.
\end{myrem}
\begin{myrem}
Observe that $(\mathbb{Z}_m,\cdot)$ is \textbf{not} a group. Most elements $a$ do not have an inverse such that $aa^{-1} = 1$ mod m.
\end{myrem}
\noindent
Actually, in cryptography it turns out that the groups playing a significant role are those with a finite number of elements. We briefly focus now on one of them, $(\mathbb{Z}_m^{*},\cdot)$, the multiplicative group of $\mathbb{Z}_m$.\\
Let's start with some definitions.
\begin{mydef}
    Given $x,y \in \mathbb{Z}$, $gcd(x,y)$ is the greatest common divisor of $x,y$.
\end{mydef}
\label{defA2}
\begin{myrem}
    If $gcd(x,y)=1$, we say that $x,y$ are relatively prime.
\end{myrem}
\begin{mylemma}
    $\forall x,y \in \mathbb{Z}$, $\exists a,b \in \mathbb{Z}$ s.t. $ax+by=gcd(x,y)$. a,b, can be efficiently found through the extended Euclidean algorithm (see \cite{UnderstandingCrypto} for details).
\end{mylemma}
\noindent
Given the definition of inverse element of a group seen above, we introduce the following:
\begin{mylemma}
    $x$ in $(\mathbb{Z}_m,\cdot)$ has an inverse $\longleftrightarrow$ $gcd(x,m)=1$. 
\end{mylemma}
\begin{proof} ($\longrightarrow$) Suppose by contradiction that $gcd(x,m)>1$. Then, $\forall a: gcd(ax,m)>1$ $\rightarrow$ $ax \neq 1$ in $\mathbb{Z}_m$, which (according to \ref{defA1}) contradicts the hypothesis.\\
($\longleftarrow$) $\exists a,b$: $ax + \cancel{bm} = 1$ ($bm = 0$ mod $m$, thus $bm=0$ in $\mathbb{Z}_m$) $\rightarrow$ $ax=1$ in $\mathbb{Z}_m$ $\rightarrow$ $x$ is invertible in $\mathbb{Z}_m$, the inverse being $x^{-1}=a$.
\end{proof}
\begin{myprop}
    $(\mathbb{Z}_m^{*},\cdot)$ = $\{x \in \mathbb{Z}_m: gcd(x,m)=1\}$ forms an abelian group. The identity element is $e=1$.
\end{myprop}
\begin{proof}
    The proof is straightforward and comes from the verification of the group properties described above.
\end{proof}
\begin{myrem}
    In particular, if $m$ is prime, then $\mathbb{Z}_m^{*}$ = $\mathbb{Z}_m \backslash \{0\}$.
\end{myrem}
\subsection{Cyclic groups}
Next step before coming to field structures is to introduce the notion of \textit{cyclic group}, necessary in turn to introduce the Generalized Discrete Logarithm Problem, which is at the basis of ECC.\\
Let's first start with some preliminary definitions.
\begin{mydef}
    A group $(\mathbb{G},\circ)$ is finite if it has a finite number of elements. 
\end{mydef}
\begin{mydef}
    The order $ord(x)$ of an element $x$ of a group $(\mathbb{G},\circ)$ is the smallest positive integer $k$ such that $x^k=1$, $e=1$ being the identity element of $\mathbb{G}$.
\end{mydef}
\begin{myexample}
    In $(\mathbb{Z}_7^{*},\cdot)$, $ord(x=2)=3$. Indeed $2^1$ = $2$ mod $7$, $2^2$ = $4$ mod $7$, $2^3$ = $1$ mod $7$.\\
    It is even interesting to see that by keeping on computing powers of $x=2$, those will keep on running through the above sequence. Indeed, $2^4$ = $2$ mod $7$, $2^5$ = $4$ mod $7$, $2^6$ = $1$ mod $7$.
\end{myexample}
\noindent
Based on this, 
\begin{mydef}
    A group $(\mathbb{G},\circ)$ which contains an element $x$ with maximum order $ord(x)$ = $|\mathbb{G}|$ is said to be cyclic. Elements with maximum order are called generators and are denoted by $g$.
\end{mydef}
\begin{myrem}
\label{generator_spanning}
    The reason for which $g$ is called generator is that it generates (or spans) the entire group (all the group elements can be recovered by raising $g$ to the powers $1,\dots,|\mathbb{G}|$). In the previous example, $g=3$ is a generator.
\end{myrem}
\begin{myrem}
    It is also clear that not every element is a generator of the cyclic group ($x=2$ being an example).
\end{myrem}
\noindent
Then we introduce three more theorems defining fundamental properties of cyclic groups. The first one is the following, by Euler.
\begin{mytheorem}
    For every prime $p$, $(\mathbb{Z}_p^{*},\cdot)$ is an abelian finite cyclic group.
\end{mytheorem}
\begin{mytheorem}
\label{thm::order_divides_cardinality}
    Let $\mathbb{G}$ be a finite cyclic group. Then $\forall x \in \mathbb{G}$, $ord(x)$ divides $|\mathbb{G}|$.
\end{mytheorem}
\noindent
Consequently, in a cyclic group there exist only element orders dividing exactly the cardinality of the group. 
\begin{myrem}
    In the already analyzed $(\mathbb{Z}_7^{*},\cdot)$, the only possible element orders are $ord(x)=1,2,3$ being $|\mathbb{G}| = 6$.
\end{myrem}
\begin{mytheorem}
\label{thm::prime_order_all_gen}
    Let $\mathbb{G}$ be a finite cyclic group. Then, if $|\mathbb{G}|$ is prime, all elements $x \neq 1 \in \mathbb{G}$ are generators.
\end{mytheorem}
\begin{proof}
    By \ref{thm::order_divides_cardinality}, being the group cardinality a prime, the only possible element orders are $ord(x)=1$ or $ord(x)=|\mathbb{G}|$. As $ord(x)=1$ $\longleftrightarrow$ $x=1$, then any $x=g \neq 1 \in \mathbb{G}$ is a generator.
\end{proof}
\noindent
The last needed definition concerning group is the one of \textit{subgroups}, basically non-empty subsets $\mathbb{H}$ of a (cyclic) group $\mathbb{G}$ being themselves groups.
\begin{mytheorem}
    Let $(\mathbb{G},\circ)$ be a cyclic group. Then every element $x \in \mathbb{G}$ with $ord(x)=s$ is the generator of a cyclic subgroup with $s$ elements.
\end{mytheorem}
\section{Fields}
We can eventually introduce the definition of \textit{field}.
\begin{mydef}
    A field $K\neq0$ is a set of elements with the following properties:
    \begin{itemize}
        \item $(K,+)$ is an abelian (additive) group, with identity element $e=0$.
        \item $(K\backslash\{0\},\cdot)$ is an abelian (multiplicative) group, with identity element $e=1$.
        \item The distributivity law holds, i.e., $\forall$ $a,b,c \in K$: $a \cdot (b+c)$ = $(a\cdot b) + (a\cdot c)$.
    \end{itemize}
\end{mydef}
\begin{myexample}
    The set $(\mathbb{R},+,\cdot)$ of real numbers is a field with identity element $e=0$ for the additive group and identity element $e=1$ for the multiplicative group. $\forall a$ $\exists b = -a$ additive inverse, $\forall a\neq 0$ $\exists b =\frac{1}{a}$ multiplicative inverse.
\end{myexample}
\begin{myexample}
\label{prime_finite_field}
    For every prime $p$, the set $(\mathbb{Z}_p,+,\cdot)$ is a field.
\end{myexample}
\begin{myrem}
    \ref{prime_finite_field} shouldn't be much of a surprise, given the premises. Indeed, we have already seen $(\mathbb{Z}_p,+)$ is an additive group; $(\mathbb{Z}_p\backslash\{0\},\cdot)$ in principle ($p$ generic) wouldn't be a multiplicative group, but $p$ prime implies $(\mathbb{Z}_p\backslash\{0\},\cdot)$ $\equiv$ $(\mathbb{Z}_p^{*},\cdot)$, which we have already seen being a multiplicative group.
\end{myrem}
\begin{myrem}
    \ref{prime_finite_field} is also the most common representative of prime finite field, which cryptography is most concerned on.
\end{myrem}
\subsection{Finite fields}
\begin{mydef}
    A finite field is a field with a finite number of elements (also called Galois field).
\end{mydef}
\begin{mytheorem}
    A finite field of order $q$, denoted as $\mathbb{F}_q$, only exists if $q = p^k$, $p$ prime number, $k$ positive integer.
\end{mytheorem}
\noindent
In particular, prime finite fields (\ref{prime_finite_field} being the most representative example) takes a fundamental role in dealing with the DLP, argument of the next section.
\section{Discrete Logarithm Problem}
\label{DLP}
We conclude the appendix with the Discrete Logarithm Problem, in its various formulations.\\
Consider the prime finite field $\mathbb{F}_p$ = $(\mathbb{Z}_p,+,\cdot)$, $p$ prime.\\
It turns out that the DLP can be directly explained through cyclic groups (from which our previous concern in their description). We present it in its multiplicative form (thus DLP over $(\mathbb{Z}_p^{*},\cdot)$), the additive one being simple to recover from.
\begin{mydef}
    Given the finite cyclic group $(\mathbb{Z}_p^{*},\cdot)$, a generator $g \in \mathbb{Z}_p^{*}$ and another element $y \in \mathbb{Z}_p^{*}$, the DLP is the problem of determining the integer $1 \leq k \leq p-1$ such that $g^k = y \mod{p}$. 
\end{mydef}
\begin{myrem}
    Being $g$ a generator of a finite cyclic group and $y$ another group element, $k$ must necessarily exist due to \ref{generator_spanning}.
\end{myrem}
\begin{myrem}
    $k = \log_g{y} \mod{p}$, from which its name.
\end{myrem}
\begin{myrem}
    The obvious transposition to additive groups: determining $k$ such that $kg=y$.
\end{myrem}
\noindent
Here it comes, instead, the reason of the introduction of subgroups.
\begin{myrem}
    It is often desirable to have the DLP in groups where $|\mathbb{G}|$ is prime to prevent the so called Pohlig-Hellman attack\footnote{For security consideration we refer to \cite{UnderstandingCrypto}; the only thing we point out is that the cited one is not the only possible attack (\textit{brute-force search}, \textit{Baby-step giant-step}, \textit{Pollard $\rho$ method} being other possibilities) and that the best known algorithm runs in time $O(|\mathbb{G}|^{\frac{1}{2}})$.}; being $|\mathbb{Z}_p^{*}|=p-1$ (not prime), subgroups of  $\mathbb{Z}_p^{*}$ with $|\cdot|=n$, $n < p$, $n$ prime are usually exploited.
\end{myrem}
\subsection{Generalized Discrete Logarithm Problem}
It is possible to generalize and present the DLP over an arbitrary cyclic group, which is at the basis of ECC.
\begin{mydef}
    Given a finite cyclic group $(\mathbb{G},\circ)$ such that $|\mathbb{G}| = n$, a generator $g \in \mathbb{G}$ and $y \in \mathbb{G}$, the GDLP is the problem of determining the integer $k$, $1 \leq k \leq n$ such that $y = \underbrace{g \circ g \circ \dots \circ g}_{k \quad times}$ = $g^k$.
\end{mydef}
\begin{myrem}
    The same considerations presented before hold here.
\end{myrem}